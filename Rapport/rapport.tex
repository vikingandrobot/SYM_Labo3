% -----------------------------------------------------------------------
% --- DOCUMENTS ---
% -----------------------------------------------------------------------
\documentclass[francais,12pt]{article}
\usepackage[utf8]{inputenc}
\usepackage{ae, pslatex}
\usepackage[french]{babel}
\selectlanguage{french} 

\usepackage{mathtools}
\usepackage{amssymb}
\usepackage{pgfplots}
\usepackage{caption}
\usepackage{hyperref}
\hypersetup{
	colorlinks=true,
	linkcolor=blue,
	filecolor=magenta,      
	urlcolor=cyan,
}

\usepackage{titlesec}
\usepackage{color}
\usepackage{colortbl}

\usepackage{hhline,tabu}

\frenchbsetup{StandardLists=true}

% -----------------------------------------------------------------------
% --- CODE JAVA ---
% -----------------------------------------------------------------------
\usepackage{listings} % pour afficher du code
\definecolor{mauve}{rgb}{0.472,0.035,0.218}
\definecolor{darkGreen}{rgb}{0.0429,0.601,0.0117}
\definecolor{antiFlashWite}{rgb}{0.95,0.95,0.96}


\lstdefinelanguage{Java}{
	morekeywords={typeof, new, true, false, catch, function, return, null, catch, switch, var, if, in, while, do, else, case, break, let,this},
	morecomment=[s]{/*}{*/},
	morecomment=[l]//,
	morestring=[b]",
	morestring=[b]',
	morestring=[s]{/[}{/;}
}

\lstdefinestyle{javaCode}
{
	% language related
	language=Java,
	keywordstyle=\color{blue},
	commentstyle=\color{green},
	stringstyle=\color{mauve},
	basicstyle=\footnotesize\ttfamily, % Standardschrift
	numbers=left,               % Ort der Zeilennummern
	numberstyle=\tiny,          % Stil der Zeilennummern
	stepnumber=2,               % Abstand zwischen den Zeilennummern
	numbersep=5pt,              % Abstand der Nummern zum Text
	tabsize=4,                  % Groesse von Tabs
	extendedchars=true,         %
	breaklines=true,            % Zeilen werden Umgebrochen
	keywordstyle=\color{blue}\bfseries,
	frame=b,
	showspaces=false,           % Leerzeichen anzeigen ?
	showtabs=false,             % Tabs anzeigen ?
	xleftmargin=17pt,
	framexleftmargin=17pt,
	framexrightmargin=5pt,
	framexbottommargin=4pt,
	backgroundcolor=\color{antiFlashWite},
	showstringspaces=false,      % Leerzeichen in Strings anzeigen ?
}

% -----------------------------------------------------------------------
% --- MARGES ---sp
% -----------------------------------------------------------------------
\usepackage{vmargin}
\setpapersize{A4}
\setmarginsrb{60pt}{50pt}{60pt}{25pt}{15pt}{25pt}{15pt}{25pt}

% -----------------------------------------------------------------------
% --- EN-TETE ET PIED DE PAGE ---
% -----------------------------------------------------------------------
\usepackage{fancyhdr}
\usepackage{lastpage}
\usepackage[utf8]{inputenc}
\pagestyle{fancy}

\fancyhead[L]{SYM - Systèmes mobiles}
\fancyhead[R]{IL - TIC - HEIG-VD \\ Automne 2017}
\fancyfoot[C]{\thepage{}}

\title{Systèmes mobiles \\ Laboratoire n\textordmasculine3 : Utilisation de données environnementales}
\author{Mathieu Monteverde, Sathiya Kirushnapillai, Michela Zucca}
\date{Automne 2017}

\titlespacing\section{0pt}{12pt plus 4pt minus 2pt}{0pt plus 2pt minus 2pt}
\titlespacing\subsection{0pt}{12pt plus 4pt minus 2pt}{0pt plus 2pt minus 2pt}
\titlespacing\subsubsection{0pt}{12pt plus 4pt minus 2pt}{0pt plus 2pt minus 2pt}


% ***********************************************************************
% *** DOCUMENT PRINCIPAL ***
% ***********************************************************************
\begin{document}
	
	\maketitle
	
	\setlength{\parskip}{1em}
	
	\section*{Balises NFC}
	\subsection*{Manipulation}
	\subsection*{Réponse aux questions}
	Sachant que les collaborateurs de l'entreprise UBIQOMP SA se déplacent en véhiculant des informations précieuses dans leurs dispositifs informatiques mobiles (munis de dispositifs de lecture NFC), et qu'ils sont amenés à se rendre dans des zones à risque, un expert a fait les estimations suivantes : 
	
	Enseignant: Fabien Dutoit 5 / 8 Assistants: Michaël Sandoz, Luca Bianchi 
	\begin{itemize}
		\item La probabilité de vol d'un mobile par une personne malintentionnée et capable d'utiliser les données à des fins préjudiciables pour la société est de 1\% 
		\item La probabilité que le mot de passe puisse être découvert, soit par analyse des traces de doigts sur l'écran, soit par observation en cours d'utilisation est de 4\% 
		\item La probabilité de vol du porte-clés est de 0.1\%  
		\item Environ 10\% des criminels susceptibles d'accéder aux données du mobile sait que le porteclés permet l’accès au mobile.
	\end{itemize}

	 Quelle est la probabilité moyenne globale que des données soient perdues, dans le cas où il faut la balise ET le mot de passe, ainsi que dans le cas où il faut la balise OU le mot de passe (on négligera dans le calcul la probabilité de l’intersection des deux ensembles), ou encore le cas où seule la balise est nécessaire ? En d'autres termes, si l'on envoie cent collaborateurs en déplacement, quel est le risque encouru de vol de données sensibles ? Mettre vos conclusions en rapport avec l'inconfort subjectif de chaque solution. Peut-on améliorer la situation en introduisant un contrôle des informations d'authentification par un serveur éloigné (transmission d'un hash SHA256 du mot de passe et de la balise NFC) ? Si oui, à quelles conditions ? Quels inconvénients ? Proposer une stratégie permettant à la société UBIQOMP SA d'améliorer grandement son bilan sécuritaire, en détailler les inconvénients pour les utilisateurs et pour la société. 
	 
	 {\color[rgb]{0,0.5,0.23}\textbf{Réponse}}
	 
	 \subsection*{Probabilités de perte de données}
	 Pour les sous-sections suivantes, nous désignerons les 4 événements de la facon suivante: 
	 
	 \begin{itemize}
	 	\item Soit A le fait que le mobile se fasse dérober par une personne malintentionnée
	 	\item Soit B le fait que le mot de passe soit découvert
	 	\item Soit C le fait que porte-clé soit dérobé
	 	\item Soit D le fait qu'un criminel ayant dérobé le mobile et le porte-clé sache que le porte-clé permet l'accès au téléphone
	 \end{itemize}
 
 	Soient les probabilités suivantes:
  	\begin{itemize}
  		\item P(A) = 0.01
  		\item P(B) = 0.04
  		\item P(C) = 0.1
  		\item P(D) = 0.1
  	\end{itemize}
	 
	 \subsubsection*{La balise et le mot de passe sont nécessaires}
	 
	 Les conditions suivantes sont donc nécessaires : A ET B ET C ET D
	 
	 P = $ P(A)  \cdot P(B) \cdot P(C) \cdot P(D) = 0.000004 $
	 
	 \subsubsection*{Le mot de passe ou la balise sont nécessaires}
	 Les conditions suivantes sont donc nécessaires : (A ET B) OU (A ET C ET D)
	 
	 P = $ (P(A) \cdot P(B)) + (P(A) \cdot P(C) \cdot P(D)) = 0.0005 $
	 
	 \subsubsection*{Seule la balise est nécessaire}
	 Nous intréprétons cette question comme le fait que seule la balise NFC permet l'accès aux données sensibles et qu'il n'y a pas de système de mot de passe. En effet le cas ou le mot de passe constitue un autre moyen d'accéder aux données est déjà traité dans la section précédente.
	 
	 Les conditions suivantes sont nécessaires : (A ET C ET D)
	 
	 P = $ P(A) \cdot P(C) \cdot P(D) = 0.0001 $
	 
	 \subsubsection*{Conclusion}
	 On remarque que le cas de figure le plus sécurisé (sans surprise) est celui où il est nécessaire de connaître le mot de passe, et de posséder la balise pour accéder aux données sensibles. On remarque également que le fait que soit le mot de passe ou soit la balise suffisent n'augmente pas la sécurité, mais au contraire augmente les chances de perdre des données. En effet si plusieurs mécanismes différents permettent l'accès aux données, et qu'il n'y pas de lien entre eux, on augmente les possibilités d'intrusion au sein du système sécurisé. 
	 
	 On remarque que bien que cette façon de faire soit sûrement plus ergonomique et agréable pour les collaborateurs, la sécurité en est affectée en mal.
	 
	 Finalement on notera peut-être qu'il ne faut pas trop compter sur l'ignorance des attaquants concernant l'utilisation de la balise pour estimer les risques d'intrusions. La sécurité par l'obscurité n'est jamais une valeur sûre...
	 
	 \subsection*{Ajout d'un serveur éloigné}
	 \subsubsection*{Conditions}
	 Il est possible d'améliorer un peu la sécurité en ajoutant un serveur éloigné qui contrôle les informations d'authenfication. Il faut cependant respecter certaines conditions. 
	 
	 Les données ne doivent pas rester sur le téléphone plus longtemps que nécessaire. Même si l'authentification est effectuée en terrain controlé sur le serveur, si les données sensibles restent sur le téléphone, l'attaquant pourra les consulter. Il faudrait donc que le serveur fournisse les données après une authentification réussie et qu'elles soient effacées après un court intervalle de temps (de la mémoire vive ou de la mémoire de stockage). Cela permettra donc également de bloquer les utilisateurs qui se sont fait dérober leur terminal. 
	 
	 Il faut également pouvoir identifier la source de la requête d'authenfication (s'assurer qu'il s'agit d'un terminal connu et sûr) et également essayer de protéger au mieux les données le long de la transmission.  
	 
	 \subsubsection*{Inconvénients}
	 L'inconvénient évident de ce système est que le terminal mobile nécessite d'être connecté au réseau pour pouvoir être utilisé. Il est également nécessaire que l'utilisateur entre son mot de passe et scanne le badge à chaque utilisation. Cela peut donc être une source d'inconfort pour l'utilisateur. 
	 
	 \subsection*{Stratégie}
	 L'idée serait donc de mettre en place un serveur d'authentification au sein d'un environnement controlé de l'entreprise. Cela permettrait de sécuriser l'étape d'authentifcation et d'autorisation en étant sûr que le code exécuté n'a pas été altéré par un attaquant. 
	 
	 Les données sensibles ne doivent pas rester sur le téléphone (en mémoire vive, seulement le temps de la consultation et des traitements nécessaires). Il faudrait également que les données soient effacées de l'application exécutée après un certain temps d'inactivité. Les données transmises doivent évidemment être chiffrées.
	 
	 L'utilisateur doit entrer son mot de passe à chaque utilisation, il ne doit pas être stocké en dur. 
	 Il serait également bien de changer régulièrement de paires mot de passe / balise NFC pour éviter qu'un attaquant puisse attaquer les deux séparèment sur plusieurs jours par exemple. 
	 
	 Les inconvénients pour les utilisateurs sont évidemment la perte du mode hors-ligne et les contraintes d'utilisation. Pour la société, il s'agit d'une stratégie beaucoup plus complexe à maintenir et gourmande en ressources.
	 
	\section*{Codes-barres}
	\subsection*{Manipulation}
	\subsection*{Réponse aux questions}
	Comparer la technologie à codes-barres et la technologie NFC, du point de vue d'une utilisation dans des applications pour smartphones, dans une optique : 
	\begin{itemize}
		\item Professionnelle (Authentification, droits d’accès, clés de chiffrage)
		\item Grand public (Billetterie, contrôle d’accès, e-paiement) 
		\item Ludique (Preuves d'achat, publicité, etc.)
		\item Financier (Coûts pour le déploiement de la technologie, possibilités de recyclage, etc.) 
	\end{itemize}

	{\color[rgb]{0,0.5,0.23}\textbf{Réponse}}
	
	La technologie à codes-barres présentent les avantages d'être peut onéreuse voir gratuite. Elle est facile à utiliser, présente depuis longtemps dans la société, facile à reproduire. Permet de stocker des petites informations de 1 à 40 caractères pour les codes-barres 1D, et de 5 à 3000 caractères pour les QR-Codes. 

	Cette technologie est plus adaptée à une utilisation grand public, comme l’étiquetage des marchandises, la billetterie, bons d'achats... 
	Elle peut également être utilisée sur les bulletins de versements pour contenir les informations relatives au paiement. Par exemple l'application BCV qui permet de scanner le code de la facture et saisie pour vous les informations du paiement. Toutefois, c'est à l'utilisateur de valider la transaction.
	
	Pour lire des codes-barres, il est nécessaire de posséder une application tierce pour la lecture et le décodage du code. Il faut évidemment cadrer correctement le code, ce qui peut prendre quelques secondes.
	
	En résumé cette technologie ne doit pas contenir d'informations sensibles et permet la lecture de données pour faciliter et accélérer une tierce fonction. Elle doit être associée à une application, si l'on souhaite que les informations enregistrés puissent aboutir à de réelles actions.
	
	La technologie NFC présente les avantages de ne pas nécessiter une tierce application pour son utilisation et de stocker un nombre important de données (jusqu'à quelque Ko). Contrairement à la technologie codes-barres, NFC ne peut pas être partiellement ou totalement lu à l'œil nu. 
	
	Cette technologie est utilisée dans un contexte professionnel, authentification, droit d'accès. Par exemple les imprimantes de la HEIG-VD utilisent un badge pour authentifier l'utilisateur souhaitant se connecter à une imprimante. Ces badges utilisent la technologie NFC. Cette technologie est également utilisée pour le paiement sans contact. Lorsque l'on approche le natel ou la carte muni d'une puce, l'on peut déclencher une action, comme un paiement, une validation.	
	Il suffit de posséder un téléphone portable pour pouvoir scanner le contenu d'une étiquette NFC. 
	
	Toutefois, la production d'étiquettes NFC est plus couteuses que la production de codes-barres et pour une utilisation simple (tel que le prix d'un article) elle n'est pas avantageuse. 
	
	Cette technologie est actuellement encore crainte par les utilisateurs, notamment à cause des soucis liés à la sécurité. 	
	
	En résumé, pour une utilisation grand public il est conseillé d'utiliser la technologie codes-barres tant que celle-ci suffit. Si l'on souhaite authentifier ou proposer des services plus complexes alors on passera sur NFC.
		
	\begin{tabular}{|l|p{6cm}|p{6cm}|}
		\hline 
		\rowcolor{lightgray}& Codes-barres (2D) & NFC \\ 
		\hline 
		Technologie & Code 2D  & RFID  \\ 
		\hline 
		Portée & 5-10m  & Jusqu'à 5m \\ 
		\hline 
		Compatibilité & Tout smartphone  & Smartphones équipés NFC  \\ 
		\hline 
		Utilisation & Application de scann & Sans application \\ 
		\hline 
		Cout &  très faible/gratuit & 20cts à 2euros par Tag \\ 
		\hline 
		Avantages & Simple à déployer sur tout support, à moindre cout. Technologie connue, universelle.  & Aucune application requise. Simplicité et rapidité d'utilisation pour l'utilisateur. \\ 
		\hline 
		Désavantages & Expérience utilisateur limitée. Lecture difficile en cas de QR Code altéré ou mal placé. & Exclus les utilisateurs apple (paiement NFC). Usage encore peu connu \\ 
		\hline 
	\end{tabular} 
	
	\section*{Balises iBeacon}
	\subsection*{Manipulation}
	\subsection*{Réponse aux questions}
	Les iBeacons sont très souvent présentés comme une alternative à NFC. Pouvez-vous commenter cette affirmation en vous basant sur 2-3 exemples de cas d’utilisations (use-cases) concrets (par exemple epaiement, second facteur d’identification, accéder aux horaires à un arrêt de bus, etc.). 

	{\color[rgb]{0,0.5,0.23}\textbf{Réponse}}
	
	Les iBeacons permettent d'envoyer des informations à des personnes à proximités de la balise. Cette technologie utilise le bluetooth pour communiquer avec les utilisateurs. Cette technologie permet des actions très proches de NFC. Comme pour la technologie codes-barres les iBeacons nécessite une application tierce pour pouvoir décoder les informations.
	
	Les informations contenus dans un iBeacon sont très petites, généralement se seront des ids qui seront interprétés par l'application et qui sera dès lors quoi faire.
	
	iBeacon permet de détecter une proximité immédiate mais également à plus grande distance. Elle couvre l'utilisation faite par NFC mais ouvre également de nouveaux usages. Par exemple une voiture équipée, détecterait si l'usager s'éloigne sans avoir verrouiller la voiture et pourrait dès lors le faire, au contraire le voyant s'approcher la déverrouiller.
	
	Les balises BLE ne sont pas prévues pour faire du paiement, contrairement à NFC. A noter également qu'une balise iBeacon est bien plus onéreuse qu'un tag NFC.
	
	Il existe 3 notions de régions pour les iBeacons
	\begin{itemize}
		\item région FAR (2à 100m) localisation imprécise. Elle peut être utilisée par exemple pour :
		\begin{itemize}
			\item Une message d'accueil
			\item Du CRM via iBeacon (fiche du client qui s'affiche sur l'ordinateur de la réceptionniste)
			\item Coupons de réductions
		\end{itemize}
		\item région NEAR (50cm à 2m) la position de l'utilisateur est connue et précise. Elle peut être utilisée par exemple pour :
		\begin{itemize}
			\item geo-marketing (push d'informations sur un produit, un tableau, une œuvre..)
			\item micro-localisation d'un utilisateur (dans une grande surface, une salle de concert pour guider le public vers sa place, une gare, un aéroport..)
			\item affichage de la fiche d'un patient sur la tablette du médecin lorsqu'il s'approche de la chambre
		\end{itemize}
		\item région IMMEDIATE ( < 50cm) la localisation est très précise et on retrouve les usages dédiés à NFC.
	\end{itemize}

		Toutefois, l'un des grands désavantages des iBeacons est que leurs signaux sont facilement perturbés. Une simple main posée sur la balise suffit à cacher le signal. 
	
		En résumé, les iBeacons pourront peut-être se développer dans des domaines que NFC ne peut pas couvrir (localisation à longues distances). Mais concernant les paiements sans contacts ou les authentifications il y a peu de chance que cela change pour l'instant.
		L’une des premières raisons étant le coût des balises comparé à celui des tags NFC. Mais également parce qu’il faut du temps pour qu’une technologie soit acceptée dans des milieux sensibles (transactions d’argents, identifications..) et il faut que la nouvelle technologie est mieux à proposer pour entrer en concurrence sur le domaine. Actuellement iBeacon ne propose pas des fonctionnalités supplémentaires ou améliorées sur les ondes à courtes distances, dès lors NFC n’a aucune raison d’être remplacé.
	
	
		\begin{tabular}{|l|p{6cm}|p{6cm}|}
		\hline 
		\rowcolor{lightgray}& iBeacon & NFC \\ 
		\hline 
		Technologie & Bluetooth Low Energy  & RFID  \\ 
		\hline 
		Portée & jusqu'à 50m  & Jusqu'à 5m \\ 
		\hline 
		Compatibilité & Smarphones équipés BLE  & Smartphones équipés NFC  \\ 
		\hline 
		Utilisation & Application propriétaire & Sans application \\ 
		\hline 
		Cout &  15 à 50 euro par balise & 20cts à 2euros par Tag \\ 
		\hline 
		Avantages & Permet d'envoyer des informations ciblés à un nombre illimités d'utilisateurs en fonction de leur position.  & Aucune application requise. Simplicité et rapidité d'utilisation pour l'utilisateur. \\ 
		\hline 
		Désavantages & S'adresses uniquement aux utilisateurs de l'application dédiée. Caractère possiblement intrusif. Balise à durée de vie limitée. & Exclus les utilisateurs apple (paiement NFC). Usage encore peu connu \\ 
		\hline 
	\end{tabular} 
	
	\section*{Capteurs}
	\subsection*{Manipulation}
	\subsection*{Réponse aux questions}
	Une fois la manipulation effectuée, vous constaterez que les animations de la flèche ne sont pas fluides, il va y avoir un tremblement plus ou moins important même si le téléphone ne bouge pas. Veuillez expliquer quelle est la cause la plus probable de ce tremblement et donner une manière (sans forcément l’implémenter) d’y remédier. 
	
	{\color[rgb]{0,0.5,0.23}\textbf{Réponse}}
	
	\subsection*{Cause probable de ce phénomène}
	La cause la plus probable de ce tremblement est la présence de bruits parasites dans les données calculées par les différents capteurs, tant au moment de la réception des données livrées par les capteurs d'accélération et de magnétisme, qu'au moment du calcul de la matrice de rotation avec la méthode \textbf{getRotationMatrix} de la classe SensorManager. 
	
	En effet, les phénomènes physiques ne sont pas des événements ayant des propriétés constantes. Nous ne sommes pas physiciens, mais nous ne serions pas surpris qu'un champ magnétique perçu par l'appareil oscille avec le temps même si le téléphone ne bouge pas. 
	
	De petites variations sont donc visibles a l'oeil nu sur la rotation de de la flèche de la boussole.
	
	\subsection*{Solution}
	En faisant des recherches sur internet, notamment sur des exemples d'implémentation de boussole pour Android justement, nous avons découvert la notion de filtre passe-bas. 
	
	Ces filtres permettent (dans ce contexte) de réduire les petites fluctuations des valeurs fournies par les capteurs. Le listing \ref{filtre_passe_bas} présente un exemple de filtre passe-bas que nous avons trouvé : 
	
	\lstset{escapechar=@,style=JavaCode}
	\begin{lstlisting}[caption={Exemple de filtre passe-bas},label={filtre_passe_bas}]
	protected float[] lowPass( float[] input, float[] output ) {
		if ( output == null ) {
			return input;  
		}
		   
		for ( int i=0; i<input.length; i++ ) {
			output[i] = output[i] + ALPHA * (input[i] - output[i]);
		}
		return output;
	}
	\end{lstlisting}
	
	Dans cette méthode, le paramètre input représente les nouvelles valeurs et output les anciennes valeurs des capteurs. La valeur de la constante ALPHA permet de spécifier la force du filtre. Lorsqu'elle vaut 0, les valeurs ne changent plus (on garde toujours output). Quand elle vaut 1, le filtre n'a pas d'effet.
	
	En regardant le code, on remarque que les petites variations de valeurs seront donc atténuées plus fortement que les grandes, et donc cela permet de réduire les perturbations.
	
	Nous avons tenté d'implémenter cet algorithme en l'appliquant à la matrice de rotation calculée pour afficher la boussole. Nous avons constaté que les tremblements étaient en effet atténués (avec par exemple ALPHA = 0.1). La boussole ne tremblait plus du tout lorsque le téléphone ne bougeait plus, et fonctionnait correctement. Cependant, nous avons eu d'autres problèmes d'affichages probablement liés au fait que nous réutilisons les deux matrices de rotations entre les affichages (et donc nous modifions parfois des matrices utilisées dans l'affichage). Nous avons donc décidé de ne pas retenir cette amélioration dans le code rendu car nous n'avons pas eu le temps de la perfectionner.
	
	En conclusion, cette solution semble fonctionner, mais elle possède sans doute un coût de traitement. Il faut en effet faire attention de ne pas modifier les valeurs utilisées par l'affichage avant d'avoir appliqué le filtre passe-bas. Cela demande donc d'effectuer des copies de valeurs et d'utiliser des matrices supplémentaires. 
	
	\subsubsection*{Sources}
	Exemple de boussole Android utilisant un filtre passe-bas
	
	https://www.ssaurel.com/blog/learn-how-to-make-a-compass-application-
		for-android/
	
	Exemple d'implémentation de filtre passe-bas
	
	https://www.built.io/blog/applying-low-pass-filter-to-android-sensor-s-readings
	
	
\end{document}

